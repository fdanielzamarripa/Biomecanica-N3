\documentclass{article}
\setlength{\parskip}{5pt} % esp. entre parrafos
\setlength{\parindent}{0pt} % esp. al inicio de un parrafo
\usepackage{amsmath} % mates
\usepackage[sort&compress,numbers]{natbib} % referencias
\usepackage{url} % que las URLs se vean lindos
\usepackage[top=25mm,left=20mm,right=20mm,bottom=25mm]{geometry} % margenes
\usepackage{hyperref} % ligas de URLs
\usepackage{graphicx} % poner figuras
\usepackage[spanish]{babel} % otros idiomas
\usepackage[utf8]{inputenc}
\author{Juan Javier Missael Castillo Ruiz y Felipe Daniel Zamarripa valdez} % author
\title{Introduccion a la Biomecanica N3} % titulo
\date{\today}

\begin{document} % inicia contenido

\maketitle % cabecera

\begin{abstract} % resumen
Podemos definir la biomecánica como una disciplina que estudia el movimiento del cuerpo en sus diferentes circunstancias es decir, esta ciencia trata de analizar la actividad del ser humano y la respuesta que tiene nuestro organismo ante esto. Dicho de esta manera, puede que se piense que la biomecánica no tiene sentido alguno pero mucho más allá esta disciplina ha ayudado a resolver muchos problemas del cuerpo humano. La persona es movimiento. Siempre estamos haciendo actividades que consisten en ir de un lado para otro o de estar moviéndonos según lo requiera aquello en lo que andamos inmersos en este momento. En cualquier caso, movernos nos sirve para estar en contacto con las cosas o para a nuestro punto de destino.
\end{abstract}

\section{Introducci'{o}n}\label{intro} % seccion y etiqueta


 Para comprender que es cabalmente la biomecánica hay que partir de una correcta definición de la física: "la física se ocupa de los fenómenos físicos; es decir, de aquellos que no modifican la estructura íntima de la materia, a diferencia de los químicos, que si la modifican".

Las cosas más simples y cotidianas están gobernadas por las leyes de la física: el movimiento de una puerta, el caminar, el correr, etc. La física esta en todas partes y sus leyes fundamentales se hacen evidentes aun en las cosas más comunes.

A medida que fue evolucionando, la ciencia física se dividió en distintas ramas: mecánica, calor, sonido, electricidad, magnetismo, óptica y física nuclear. Pero la que incumbe específicamente a la biomecánica es la primera, que estudia los movimientos de los cuerpos y las fuerzas que en ellos actúan. A su vez la biomecánica se podría definir como el estudio mecánico de un medio biológico. En este caso ese medio seria el cuerpo humano, particularmente.

A fin de estudiar los movimientos del cuerpo humano, se parte de un modelo mecánico que se acerque a la realidad en su máximo posible: el sistema locomotor esta constituido principalmente por una estructura ósea y otra muscular encargada de mover esos segmentos óseos. Elegir un modelo que represente este sistema es bastante complejo. Se le requiere simpleza para facilitar la comprensión a la vez que las suficientes cualidades que permiten obtener resultados validos. Con este objeto resulta útil el modelo óseo que se adoptara en 1955.

Los segmentos óseos se representan como rectas que pasan por los centros de las superficies articulares situadas en el extremo de los huesos a las que técnicamente se denomina " Eje Mecánico ".    De esta manera se forman eslabones que permiten considerar a los movimientos humanos como acciones angulares cuya libertad se limita por las posibilidades de las articulaciones.
\clearpage

\section{Desarrollo}

 La \textbf {biomecánica} es una disciplina científica que tiene por objeto el estudio de las estructuras de carácter mecánico que existen en los seres vivos (fundamentalmente del cuerpo humano). Esta área de conocimiento se apoya en diversas ciencias biomédicas, utilizando los conocimientos de la mecánica, la ingeniería, la anatomía, la fisiología y otras disciplinas, para estudiar el comportamiento del cuerpo humano y resolver los problemas derivados de las diversas condiciones a las que puede verse sometido.
 
 \hspace{1cm}

\begin{minipage}{.89\linewidth} % figura
\centering
\includegraphics[width=80mm]{R.jpg} % archivo
\end{minipage}
 
\hspace{1cm} 

La biomecánica está íntimamente ligada a la biónica y utiliza algunos de sus principios ha tenido un gran desarrollo en relación con las aplicaciones de la ingeniería a la medicina, la bioquímica y el medio ambiente, tanto a través de modelos matemáticos para el conocimiento de los sistemas biológicos como en lo que respecta a la realización de partes u órganos del cuerpo humano y también en la utilización de nuevos métodos diagnósticos.



\hspace{1cm}

\begin{minipage}{.89\linewidth} % figura
\centering
\includegraphics[width=90mm]{biomeccanica medica.jpg} % archivo
\end{minipage}

\hspace{2cm}

Una gran variedad de aplicaciones incorporadas a la práctica médica; desde la clásica pata de palo a las sofisticadas prótesis ortopédicas con mando mio-eléctrico y de las válvulas cardiacas a los modernos marcapasos existe toda una tradición e implantación de órganos artificiales.
Hoy en día es posible aplicar con éxito, en los procesos que intervienen en la regulación de los sistemas modelos matemáticos que permiten simular fenómenos muy complejos en potentes ordenadores, con el control de un gran número de parámetros o con la repetición de su comportamiento.

\clearpage

\textbf{SUB-DISCIPLINAS}
\hspace{1cm}

La Biomecánica está presente en diversos ámbitos, aunque tres de ellos son los más destacados en la actualidad:

  La \textbf {biomecánica médica}, evalúa las patologías que aquejan al cuerpo humano para generar soluciones capaces de evaluarlas, repararlas o paliarlas.Usa la simulación que es la aceleración de la forma en que las empresas y los dispositivos médicos mueven los productos a través de diferentes fases de desarrollo. Los prototipos virtuales juegan un papel fundamental en el diseño de verificación y validación. A través del prototipo virtual, el diseño puede ser verificado en contra de las especificaciones del cliente. Adicionalmente, el diseño de la Empresa pueda ser validada contra las normas reglamentarias pertinentes. El resultado final es una muy fiable y rentable diseño elaborado y validado en menos tiempo.
  



\begin{minipage}{.89\linewidth}
\centering
\includegraphics[width=5cm]{usain_bolt.jpg}
\end{minipage}

La \textbf {biomecánica deportiva} , analiza la práctica deportiva para mejorar su rendimiento, desarrollar técnicas de entrenamiento y diseñar complementos, materiales y equipamiento de altas prestaciones.El objetivo general de la investigación biomecánica deportiva es desarrollar una comprensión detallada de los deportes mecánicos específicos y sus variables de desempeño para mejorar el rendimiento y reducir la incidencia de lesiones. Esto se traduce en la investigación de las técnicas específicas del deporte, diseñar mejor el equipo deportivo, vestuario, y de identificar las prácticas que son predisponen a una lesión. Dada la creciente complejidad de la formación y el desempeño en todos los niveles del deporte de competencia, no es de extrañar que los atletas y entrenadores estén recurriendo en la literatura de investigación sobre la biomecánica aspectos de su deporte para una ventaja competitiva.



La \textbf {biomecánica ocupacional}, estudia la interacción del cuerpo humano con los elementos con que se relaciona en diversos ámbitos (en el trabajo, en casa, en la conducción de automóviles, en el manejo de herramientas, etc) para adaptarlos a sus necesidades y capacidades. En este ámbito se relaciona con otra disciplina como es la ergonomía.Últimamente se ah hecho popular y se ha adoptado la Biomecánica ocupacional que proporciona las bases y las herramientas para reunir y evaluar los procesos biomecánicas en lo que se refiera a la actual evolución de las industrias, con énfasis en la mejora de la eficiencia general de trabajo y la prevención de lesiones relacionadas con el trabajo, esta está íntimamente relacionada con la ingeniería médica y de información de diversas fuentes y ofrece un tratamiento coherente de los principios que subyacen a la biomecánica bien diseñado y ergonomía de trabajo que es ciencia que se encarga de adaptar el cuerpo humano a las tareas y las herramientas de trabajo

\hspace{1cm} 

\begin{minipage}{.89\linewidth}
\centering
\includegraphics[width=5.5cm]{caja.jpg}
\end{minipage}
\clearpage

\textbf{Protesis}



Una prótesis es una extensión artificial que reemplaza o provee una parte del cuerpo que falta por diversas razones.

Una prótesis corporal es la que reemplaza un miembro del cuerpo, cumpliendo casi la misma función que un miembro natural, sea una pierna, un brazo, un pie, una mano, o bien uno o varios dedos. Pero existen varios otros tipos de prótesis, algunas de las cuales reemplazan funciones perdidas del cuerpo, mientras que otras cumplen funciones estéticas.


Es habitual confundir un aparato ortopédico u ortesis con una prótesis, utilizando ambos términos indistintamente. Una ortesis no sustituye total ni parcialmente a un miembro, sino que reemplaza o mejora sus funciones


\hspace{1cm} 

\begin{minipage}{.89\linewidth}
\centering
\includegraphics[width=8cm]{protesis.jpg}
\end{minipage}

\hspace{1cm}

\textbf{Tipos de Protesis}

\hspace{1cm} 

Para los miembros superiores :

\hspace{1cm} 

Hay 5 tipos generales de prótesis del miembro superior:

\hspace{.5cm} 

1.Prótesis pasivas

\hspace{.5cm} 

2-Prótesis de control corporal

\hspace{.5cm} 

3-Prótesis mioeléctricas alimentadas externamente

\hspace{.5cm} 

4-Prótesis híbridas

\hspace{.5cm} 

5-Prótesis específicas para cada actividad


\hspace{1cm} 

Las \textbf {prótesis pasivas} ayudan al equilibrio, la estabilización de objetos (como un papel al escribir) o actividades recreativas/vocacionales. Parecen un miembro natural, son las más ligeras y económicas, pero no permiten la prensión activa de las manos.

\hspace{1cm} 

\begin{minipage}{.89\linewidth}
\centering
\includegraphics[width=9cm]{pasiva.jpg}
\end{minipage}


\hspace{1cm} 

\hspace{.5cm} 

Las \textbf {prótesis de control corporal}  son las indicadas con mayor frecuencia porque tienden a .ai menos costosas, más duraderas y requieren menos mantenimiento. Un sistema de cable con arnés suspende la prótesis y captura el movimiento escapular y humeral para operar la articulación del gancho, la mano o el codo. Algunos sistemas usan el brazo opuesto para activar una función particular; un extremo de una correa rodea el brazo opuesto en la axila, y el otro extremo se conecta a un cable que controla el dispositivo terminal (gancho, mano o dispositivo especializado para una función en particular). Las personas que realizan trabajo físico suelen preferir este tipo.

\hspace{1cm} 

\begin{minipage}{.89\linewidth}
\centering
\includegraphics[width=5.5cm]{control corporaal.jpg}
\end{minipage}


\hspace{1cm} 


Las \textbf {prótesis mioeléctricas alimentadas externamente} permiten movimientos activos de las manos y las articulaciones sin necesidad de movimientos escapulares, humerales o del tronco. Los sensores y otras entradas detectan el movimiento muscular del miembro residual o la parte superior del cuerpo y controlan actuadores eléctricos que proporcionan una mayor fuerza de prensión que las prótesis de control corporal.

\hspace{1cm} 

\begin{minipage}{.89\linewidth}
\centering
\includegraphics[width=5.5cm]{mioelectricas.jpg}
\end{minipage}




Las \textbf {prótesis híbridas por lo general} , se prescriben para amputaciones de las porciones proximales de los miembros superiores. Combinan características específicas de las prótesis de control corporal y las mioeléctricas, por ejemplo, un codo de control corporal podría combinarse con una mano activada por energía externa o un dispositivo terminal.

\hspace{1cm} 

\begin{minipage}{.89\linewidth}
\centering
\includegraphics[width=5.5cm]{hibrida.jpg}
\end{minipage}


\hspace{1cm} 

Las \textbf { prótesis específicas para cada actividad} están diseñadas para permitir la participación en actividades que de otro modo dañarían el miembro residual del paciente o la prótesis habitual, o en situaciones en las cuales la prótesis habitual no funcionaría de manera eficaz. Estas prótesis a menudo incluyen diseños especiales para la interfaz, el receptáculo, el sistema de suspensión y el dispositivo terminal. Los dispositivos terminales específicos para la actividad pueden permitir al paciente sostener un martillo y otras herramientas, un palo de golf o un bate de béisbol, o colocarse un guante de béisbol. Otros ayudan en diversas actividades específicas (p. ej., natación, pesca). Estos dispositivos pueden .ai pasivos o controlados por el amputado.

\hspace{1cm} 

\begin{minipage}{.89\linewidth}
\centering
\includegraphics[width=7cm]{para cada actividad.jpg}
\end{minipage}


\hspace{1cm} 


\textbf {Prótesis de mano parcial}
Las amputaciones de mano parcial varían desde un solo dedo o múltiples dedos hasta amputaciones carpometacarpianas; la flexión y la extensión de la muñeca generalmente se con.aivan. La restauración protésica funcional es posible cuando falta toda la mano o uno o más dedos utilizando energía mecánica o externa. La prensión y la oposición a menudo se pueden lograr si se oponen de alguna manera los dedos naturales y los protésicos.

\hspace{1cm} 

\begin{minipage}{.89\linewidth}
\centering
\includegraphics[width=7cm]{parcial.jpg}
\end{minipage}


\hspace{2cm} 

\textbf {Prótesis para la desarticulación de la muñeca}
La amputación de desarticulación de la muñeca elimina todos los huesos del carpo, sin dejar capacidad para flexionar o extender la muñeca. La pronación y la supinación se retienen en su mayor parte. Se puede utilizar una prótesis de mano, un gancho o un dispositivo terminal de actividad especial. Se pueden usar dispositivos pasivos, de control corporal o de control externo (mioeléctricos).

\hspace{0.5cm} 

\begin{minipage}{.89\linewidth}
\centering
\includegraphics[width=4.2cm]{desarticulado de muñeca.png}
\end{minipage}


\textbf {Desarticulación del codo y prótesis por encima del codo}
La desarticulación del codo y las prótesis por encima del codo requieren un codo mecánico. Las prótesis para la desarticulación del codo generalmente emplean la potencia del cuerpo para flexionar el codo (la gravedad extiende el codo) y el control mioeléctrico del dispositivo terminal. Dos bisagras externas en el codo están unidas al exterior del receptáculo de plástico. Hay muchas combinaciones de sistemas de codo y de control.

\hspace{1cm} 

\begin{minipage}{.89\linewidth}
\centering
\includegraphics[width=5.5cm]{encima del codo.jpg}
\end{minipage}


\clearpage

\textbf {Desarticulación del hombro y prótesis interescapular/para el cuarto delantero}
En la desarticulación del hombro y las prótesis interescapulares, la disipación de calor, la distribución del peso y la comodidad son de suma importancia. La superficie de contacto puede .ai de plástico rígido o flexible, o un material de almohadilla de gel como la silicona. Las prótesis más funcionales para estos niveles de amputación generalmente incluyen control mioeléctrico de una o más articulaciones y del funcionamiento de la mano.

\hspace{1cm} 

\begin{minipage}{.89\linewidth}
\centering
\includegraphics[width=5.5cm]{Desarticulación del hombro.jpg}
\end{minipage}

\hspace{2cm}
\textbf{Prótesis para los miembros inferiores }

\hspace{1cm} 

Existen muchas variables y opciones para las prótesis de miembros inferiores: hay 350 sistemas diferentes de pie/tobillo y 200 rodillas diferentes. El amputado y el protesista evalúan diferentes componentes de las articulaciones y los pies para determinar cuál proporciona equilibrio, seguridad, función y eficiencia de la marcha óptimos. Las selecciones pueden cambiar durante el proceso de adaptación cuando la evaluación biocinética determina la eficiencia óptima de la marcha.

\hspace{1cm} 

La mayoría de las prótesis de miembros inferiores son endoesqueléticas porque proporcionan un ajuste continuo de la alineación biomecánica. Esto permite al protesista refinar la cinemática de los componentes protésicos del pie, el tobillo y la rodilla debajo del centro de gravedad, minimizando el gasto de energía al caminar.

\hspace{3cm} 

Los \textbf {sistemas protésicos para el tobillo y el pie} pueden incluir sistemas hidráulicos que amortiguan las fuerzas de impacto; algunos se ajustan automáticamente a los cambios del ritmo. Los sistemas de tobillo/pie controlados por microprocesador regulan la función en tiempo real según la información del usuario y/o las condiciones ambientales. Algunos son mecanismos pasivos; otros proporcionan propulsión activa que reduce en gran medida los requerimientos de energía para caminar. La rotación axial u horizontal perdida por las amputaciones por encima del tobillo se puede reemplazar con unidades de torsión endosquelética; esta característica es especialmente útil en los golfistas. Los pacientes que tienen zapatos con diferentes alturas de tacón (p. ej., botas de vaquero, tacones altos) pueden elegir un tobillo protésico que se ajuste a diferentes alturas; sin embargo, los pies protésicos con altura de talón ajustable pueden no proporcionar un funcionamiento dinámico suficiente.

\hspace{1cm} 

\begin{minipage}{.89\linewidth}
\centering
\includegraphics[width=7cm]{protesis de pies y tobillo.jpg}
\end{minipage}


\hspace{3cm}

Los \textbf {sistemas protésicos de pie y rodilla específicos para deportes} ayudan a los amputados a alcanzar el más alto nivel de rendimiento físico. Algunos sistemas son eficaces para múltiples actividades deportivas y recreativas. Otros se diseñaron para eventos específicos (p. ej., carreras de velocidad, carreras de larga distancia, esquí, natación). Correr es más difícil para los amputados por encima de la rodilla que por debajo de la rodilla. El receptáculo y la suspensión son más críticos para los atletas. La atrofia muscular y la fluctuación de volumen son más comunes en los atletas y requieren ajustes de encaje más frecuentes.

\hspace{1cm} 

\begin{minipage}{.89\linewidth}
\centering
\includegraphics[width=4cm]{sistemas protésicos de pie y rodilla específicos para deportes.jpg}
\end{minipage}


\hspace{1cm}

\textbf {Prótesis de pie parcial}

Las amputaciones transmetatarsianas, de Lisfranc, de Chopart y de pie parcial de Boyd retienen la longitud normal del miembro y el pie residual proporciona una superficie de carga natural adecuada; muchos pacientes pueden pararse y caminar distancias cortas sin una prótesis. Los pacientes con amputación parcial del pie gastan menos energía caminando que aquellos con niveles de amputación más altos.

Una prótesis de silicona tipo zapatilla permite cierto movimiento del tobillo y una deambulación simple a velocidades lentas y medias. Para actividades más vigorosas (p. ej., caminar rápido, correr, subir escaleras y rampas), los pacientes pueden usar un receptáculo de plástico semirrígido que encapsula el pie y el tobillo remanentes y se extiende hasta el borde inferior de la rótula.


\hspace{1cm} 

\begin{minipage}{.89\linewidth}
\centering
\includegraphics[width=7cm]{Prótesis de pie parcial.png}
\end{minipage}


\hspace{1cm}

\textbf {Prótesis de desarticulación de tobillo de Syme}

\hspace{1cm} 

Las amputaciones de desarticulación del tobillo de Syme con.aivan una almohadilla gruesa de tejido del talón, que proporciona capacidad para soportar peso. Aunque la longitud del miembro se acorta de 7 a 9 cm, los pacientes generalmente pueden pararse y caminar distancias cortas sin una prótesis (p. ej., trasladarse dentro y fuera de la cama o silla, caminar a una habitación adyacente). Se dispone de varios sistemas de prótesis para el pie y el tobillo.

\hspace{1cm} 


Una amputación de Syme modificada reduce el extremo bulboso típico al recortar las prominencias de los maléolos tibial y peroneo. Esta modificación simplifica el ajuste de la prótesis y da como resultado una apariencia del tobillo menos voluminosa.

\hspace{1cm} 

\hspace{1cm} 

\begin{minipage}{.89\linewidth}
\centering
\includegraphics[width=5.5cm]{symes.jpg}
\end{minipage}


\hspace{1cm}


\textbf {Prótesis transtibiales (debajo de la rodilla) }
Los pacientes con una prótesis debajo de la rodilla correctamente ajustada y alineada pueden funcionar bien, a menudo sin anormalidad visible de la marcha. La capacidad y la función para caminar están limitadas principalmente por el estado preoperatorio del paciente y las condiciones comórbidas posoperatorias. La amputación que mantiene la longitud del miembro residual > 9 cm (justo debajo del tubérculo tibial) mantiene la in.aición del cuádriceps, lo que proporciona una mejor función que la amputación a través de la rodilla o por encima de ella.

\hspace{1cm}


\begin{minipage}{.89\linewidth}
\centering
\includegraphics[width=4cm]{debajo de la rodilla.jpg}
\end{minipage}

\hspace{1cm}

Un receptáculo con soporte total de la superficie con un receptáculo flexible primario y uno de retención semirrígido con manejo del volumen por vacío y suspensión mejora la comodidad, la conectividad y la simetría de la marcha. El vacío se establece mediante una bomba mecánica o eléctrica. Otros métodos de suspensión están disponibles y pueden emplearse de manera eficaz. El protesista y el paciente evalúan los diferentes componentes del pie/tobillo para identificar los componentes que proporcionan equilibrio, estabilidad y fluidez óptimos durante la marcha.


\hspace{1cm}


\textbf {Desarticulación de la rodilla y prótesis transfemorales (por encima de la rodilla) }

La amputación con desarticulación de la rodilla tiene ventajas y desventajas en comparación con la amputación a nivel transfemoral. La desarticulación de la rodilla retiene los cóndilos femorales y mejora la carga distal, lo que reduce la presión y las fuerzas de cizallamiento en el miembro residual y mejora la propiocepción. Una desventaja es que el centro de rotación de la prótesis no coincide con el centro de rotación de la rodilla contralateral, lo que afecta la función y la apariencia: al sentarse, una rodilla se extiende más que la otra. Una alternativa es usar una prótesis de rodilla medial y lateral con bisagras unidas al exterior del receptáculo, lo que puede establecer un mejor centro de rotación pero aumenta el ancho de la prótesis y dificulta el ajuste de la ropa.

\hspace{1cm} 

Los diseños de receptáculos para la desarticulación transfemoral y de rodilla pueden .ai rígidos, rígidos con paneles flexibles para permitir la expansión muscular o ajustables para adaptarse a las fluctuaciones normales del volumen durante el día. El sistema de suspensión puede incluir una correa mecánica, un pasador suspensor integrado, aspiración o vacío. Los pacientes y los protesistas deben evaluar diferentes sistemas protésicos para la rodilla y el pie/tobillo con el fin de lograr equilibrio, estabilidad y movilidad óptimos.

\hspace{1cm} 

\begin{minipage}{.89\linewidth}
\centering
\includegraphics[width=5.5cm]{Desarticulación de la rodilla y prótesis transfemorales (por encima de la rodilla).jpg}
\end{minipage}


\clearpage


\textbf {Prótesis con desarticulación de cadera y hemipelvectomía}

\hspace{1cm} 

El dos por ciento  de todas las amputaciones se encuentran en estos niveles, por lo que pocos protesistas y fisioterapeutas tienen experiencia con estos pacientes. El peso de la prótesis es considerablemente mayor que cualquier otro y también requiere mayor atención al encaje y la suspensión. Incluso con una prótesis óptima, la velocidad de la caminata es la mitad que la de las personas sin discapacidad y el gasto de energía es 80 por ciento mayor. Debido a los desafíos, la aceptación a largo plazo de la prótesis para la amputación en este nivel oscila entre 35 y 45 . Las expectativas irracionales aumentan la tasa de rechazo. Los resultados exitosos dependen de la comprensión realista de los desafíos y las limitaciones por parte de los pacientes; su grado de motivación, fuerza central, equilibrio y coordinación; y su capacidad para soportar el 100 por ciento  del peso corporal en el miembro residual que queda tras la amputación.

\hspace{1cm} 

\begin{minipage}{.89\linewidth}
\centering
\includegraphics[width=7cm]{cadera.jpg}
\end{minipage}


\hspace{1cm}



\section{Conclusiones}

\hspace{2cm}

\textbf{Juan Javier Missael Castillo Ruiz}

\hspace{1cm}

La Biomecánica es una disciplina que estudia y hace análisis físicos de los movimientos del cuerpo humano. El objetivo de la Biomecánica en las actividades deportivas es la caracterización y la mejora de las técnicas del movimiento a partir de conocimientos científicos.

\hspace{3cm}

\textbf{Felipe Daniel Zamarripa Valdez}

\hspace{1cm}

Al llevar a cabo esta tarea se pudo prufondizar los temas que se relacionan con la biomecanica, desde el luego el significado immportante y los aportes que proceden en los conocimientos de la fisica y mecanica en los cuerpos de los seres vivos. al realizar un analisis de moviento y articulaciones se puede encontrar la logica y hasta las limitaciones de los seres vivos.

\hspace{3cm}


\bibliography{bib}
\bibliographystyle{plainnat}

\clearpage

 \section{Bibliografia}


@article {
         author =  {Jan J. Stokosa },
         title = {Opciones para las prótesis de los miembros)},
         year =  2021
         month = {Enero}
         }
         
         %https://www.msdmanuals.com/es-mx/professional/temas-especiales/miembro-prot%C3%A9sico/opciones-para-las-pr%C3%B3tesis-de-los-miembros


\hspace{1cm}

         Opciones para las prótesis de los miembros - Temas especiales - Manual MSD versión para profesionales. (2022). Retrieved 22 August 2022, from https://www.msdmanuals.com/es-mx/professional/temas-especiales/miembro-prot%C3%A9sico/opciones-para-las-pr%C3%B3tesis-de-los-miembros 

\hspace{1cm}

@article {
         author =  {Rodrigo Ricardo },
         title = {¿Qué es la biomecánica? – Definición y aplicaciones)},
         year =  2020
         month = {Noviembre}
         }

         https://estudyando.com/que-es-la-biomecanica-definicion-y-aplicaciones/
         
\end{document}